%!TEX root = Thesis.tex

\chapter{Introduction}

  \begin{singlespace}
     The increasing numbers of sensor devices has increased the number of sensor-specific protocols, platforms and software. As a result various approaches have been proposed to interconnect, maintain and monitor various type of sensors\cite{6588063,bendel2013service,song2010real}. That specificly focused on a platform development, protocol definition and software architecture for the concreate user-oriented requirements and for a narrowly focused areas of usage instead of defining a common system approach. Mainly in proposed approaches was discovered such type of questions as security and privacy, easy of development and monitoring, social dimensions \cite{eggert2013sensorcloud}, that completely don't consider requirements and criteria of an end-user. Therefore the area of research of this master thesis is dedicated to define generic frontend for exploring sensor and information services.
     The folowing sections ground the motivation for the chosen research field, define the central research questions and goals of this master's thesis, and describe the overall structure of the work.
  \end{singlespace}

\section{Motivation}
     In the recent years with the technological progress in the computer science, information systems, and in particular sensor data systems, have become an essential part in daily life of the modern society. People have started to use them more often not only for manufactory, business, education but also for private reasons.
     Currently, most of the research is concerned with the protocol and middleware levels, whereas
     the potential of a generic interactive access to sensor and information services needs to be explored. 
     This involves their selection, mash-ups, and usage within a client-controlled interface.
     In this master thesis, a first web-based prototype (portal) for such services is to be
     created. Users should be able to explore not just services, but also the information provided by
     them, and eventually be led to advanced usage patterns such as the development
     of third-party applications to access the information data and real-time streams.

\section{Research Questions and Goals}
     Creating composite third-party services and applications from reusable components 
     is an important technique in software engineering and data management. Although a large body
     of research and development covers integration at the data and application levels,
     weak work has been done to facilitate it at generic level. This master thesis
     discusses the existing user interface frameworks and component technologies
     used in presentation integration, illustrates their strengths and weaknesses, and
     presents some opportunities for future work.

     As mentioned in the previous section, there are already exist many solutions
     for creating sensor-aware applications. But these platforms focuses on a single area of usage and they are not commonly suitable to support the dynamic and adaptable composition and usage of different type of sensors in one portal.

\begin{itemize}
\item Concept for a generic information and sensor service portal
\item Development of the portal and associated dependency tools
\item Demonstration using a convincing scenario
\end{itemize}

     Therefore, this thesis is aimed at the development of a concept that provides users a possibility to personalize their current environment indepently from any type and kind of devices.

\section{Structure}

The thesis is structured in the following way:

\emph{Chapter 2} defines the background of the master’s thesis
     describing the basic used terminology and the foundation platforms. A reference scenario and the requirements to a concept that has to be developed are also introduced in this chapter.

\emph{Chapter 3} is devoted to the state of the art analysis. The related research
     works in the areas of the sensor-driven platforms, the component based
     groupware systems, the browser based and non-browser based systems
     are investigated and evaluated against the defined requirements.

\emph{Chapter 4} focuses on the concept of the generic Frontend for exploring sensor and Information 
     sevices, considering possible approaches, strategies, frameworks and necessary criteries, defined in \emph{Chapter 3}.

\emph{Chapter 5} provides the implemented functionalities of the concept and describes evaluaion of results.

\emph{Chapter 6} concludes the master’s thesis underlining the achieved goals
     and providing prospects for the future work.
