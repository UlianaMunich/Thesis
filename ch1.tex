%!TEX root = Thesis.tex

\chapter{Introduction}

  \begin{singlespace}
     The increasing numbers of sensor devices has increased the number of sensor-specific protocols, platforms and software.  The definition "sensor" consists not only a phisical device that can be an actuator or measuring device, but also a contest-depending information that are available through the Internet, e.g. Facebook, RSS Feeds, Weather Forecast and etc. As a result various approaches have been proposed to interconnect, maintain and monitor various type of data sources\cite{6588063,bendel2013service,song2010real}.That specifically focused on a platform development, protocol definition and software architecture for the concrete user-oriented requirements and for a narrowly focused areas of usage instead of defining a common system approach. Mainly in proposed approaches was discovered such type of questions as security and privacy betwenn sensor's protocol and platform, platform development itself, online monitoring of different data sources, social dimensions\cite{eggert2013sensorcloud}, that almost not focused on requirements and criteria of a user-friendly interface, that for user nowadays become one of the mnost important thing. Therefore the area of research of this master thesis is dedicated to define generic frontend for exploring data sources, thwt can be easily and universaly interconnected to an any platform or system without any sort of Graphical User Interface(GUI). The following sections ground the motivation for the chosen research field, define the central research questions and goals of this master's thesis, and describe the overall structure of the work.
  \end{singlespace}

\section{Motivation}
     In the recent years with the technological progress in the web technologies, information systems, and in particular sensor data systems, have become an essential part in daily life of the modern society. More and more aspects of human life is shifted to the Web. This allows fast and easy way to get information from any point on earth. In the same time increased number of platform and interfaces. People have started to use them more often not only for manufacture, business, education but also for private reasons. Currently, most of the research is concerned with the protocol and middleware levels, whereas the potential of a generic interactive access to sensor and information services needs to be explored in a best comfortable way for user and developer also. In this master thesis, a first universal web-based frontend with a main focus on a GUI is to be created. Users should be able to explore not just the information provided by Web services, but also from a real sensors around them. The system architecture of such a concept should be distributed in such a way, that already implemented projects could easily interconnect with it.

     Creating composite third-party services and applications from reusable components is an important technique in software engineering and data management. Although a large body of research and development covers integration at the data and application levels, weak work has been done to facilitate it at generic level. This master thesis discusses the existing user interface frameworks and component technologies used in presentation integration, illustrates their strengths and weaknesses, and presents some opportunities for future work.

\section{Addressed Use Cases}
     In order to define concrete research questions, it is important to determine which kind of concept is going to be implemented as well as to clarify the common terms which are used throughout the thesis:
     \begin{itemize}
          \item \textbf{Application class.} This thesis is focused on a general multi-tier web-based prototype that delivers dynamic content to end-users via RESTful API.

          \item \textbf{Platform.} Implies a virtualized hardware architecture and a software stack required for running an application of the respective class.

          \item \textbf{Infrastructure.} Assumes a set of web-based protocols and sensors streaming resources used to maintain services of the aforementioned application class.
     \end{itemize}


\section{Research Questions and Goals}
       As mentioned in the previous section, there are already exist many solutions for creating sensor-aware applications. But these platforms focuses on a single area of usage and they are not commonly suitable to support the dynamic and adaptable composition and usage of different type of sensors in one dashboard.

       Within this thesis the following research questions should be answered in order to design, implement and evaluate the generic frontend for retrieving data sources: 
       \begin{itemize}
       \item Which architecture should the generic frontend have?
       \item What type of data sources are have to be retrieved and what is the universal interface for collaboration between backend and frontend system?
       \item How can be the general GUI designed and implemented?
       \item Which software components might be applied to the concept in order to be applicable to the most available data sources and platform?
       \end{itemize}

     Therefore, this thesis is aimed at the development of a concept that provides users a possibility to personalize their current environment indepently from any type and kind of devices.

\section{Structure}

After the introduction in \emph{Chapter 1} the thesis is structured in the following way:

\emph{Chapter 2} defines the background of the master’s thesis describes the basic used terminology and the foundation platforms. A reference scenario and the requirements to a concept that has to be developed are also introduced in this chapter.

\emph{Chapter 3} is devoted to the state of the art analysis. The related research works in the areas of the sensor-retrieving approaches, the component based groupware systems, the browser based and non-browser based systems are investigated and evaluated against the defined requirements.

\emph{Chapter 4} focuses on the concept of the generic Frontend for exploring sensor and Information services, considering possible approaches, strategies, frameworks and necessary criteria, defined in \emph{Chapter 3}.

\emph{Chapter 5} provides the implemented functionality of the concept and describes evaluation of results. Important methods and libraries and interfaces are described in details.

\emph{Chapter 6} concludes the master’s thesis underlining and evaluating achieved goals and providing prospects for the possible future work. Two test cases are proposed, described and performed in order to cover the fulfillment of the defined requirements by the developed solution. Afterwards the results of the evaluation are discussed.
