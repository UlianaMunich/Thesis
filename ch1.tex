%!TEX root = Thesis.tex

\chapter{Introduction}

  \begin{singlespace}
     The increasing numbers of sensor devices has increased the number of sensor-specific protocols, platforms and software. The definition "sensor" consists not only a physical device that can be an actuator or measuring device, but also a context-depending information that are available through the Web, e.g. Facebook, Tweets, RSS Feeds, Weather Forecast and etc as a real-time data streaming. As a result various approaches have been proposed to interconnect, maintain and monitor various type of data sources\cite{6588063,bendel2013service,song2010real}.That specifically focused on a platform development, protocol definition and software architecture for the concrete user-oriented requirements and for a narrowly focused areas of usage instead of defining a common system approach. Mainly in proposed approaches was discovered such type of questions as security and privacy between sensor-specific protocol and platform, platform development itself, online aggregation and monitoring of different data sources, social dimensions\cite{eggert2013sensorcloud}, that do not cover criteria of a user-friendly interface, which is for user nowadays an essential part of opportunity to explore a system. Therefore the area of research of this master thesis is dedicated to define common approach of a generic frontend together with a Graphical User Interface(GUI) for exploring data sources, that can be easily and universally interconnected to any platform or a system, which has no GUI. The following sections ground the motivation of the chosen research field, define the central research questions and goals of this master thesis, and describe the overall structure of the work.
  \end{singlespace}

\section{Motivation}
     In the recent years with the technological progress in the information systems, web technologies and in particular sensor data systems, have become an essential part in daily life of the modern society. More and more aspects of human life are shifted to the Web and mobile applications. This allows fast and easy way to get information from any point on earth by using any web, mobile, traditional desktop clients. In the same time have increased a range of sensor specific platform and interfaces. People have started to use them more often not only for manufacture, business, education but also for private reasons. Currently, most of researches is concerned with the protocol and middleware levels, whereas the potential of a generic interactive access to sensor and information services needs to be explored in a best comfortable way for a third-party user, for developer of a such system itself and future application consumer developer. As was already mentioned, most of the works in this area are focused mainly only on a backend side, thus an essential goal fo this thesis is to provide simple and universal for integration web-based generic frontend with a main focus on a dynamic GUI generation. Users and developers should be able to explore not only information provided by Web services, but also from a real sensors around them. The system architecture of such a concept should be distributed in such a way, that already implemented projects could easily interconnect with it.

     Creating composite third-party services and applications from reusable components is an important technique in software engineering and data management. Although a large body of research and development cover integration at the data and platform levels, weak work has been done to facilitate it at generic level. This master thesis discusses the existing web frameworks and component technologies used on presentation level integration, illustrates their strengths and weaknesses, and presents some opportunities for future work. Real-time web applications are connection-oriented applications, with web-based user interfaces, that display information as soon as it was published. Examples include social news aggregators and monitoring tools that continually update themselves with data from an external source.

\section{Application Area}
     In order to define concrete research questions, it is important to determine which kind of concept is going to be proposed and implemented, as well as to clarify the common terms which are used throughout the thesis:
     \begin{itemize}
          \item \textbf{Application.} This thesis is focused on a generic multi-tier web-based frontend that delivers dynamic real-time content to end-users via user-friendly graphical interface. Such type of universal frontend acts as middleware between provider-specific data source and a user. By supporting a concept architecture any backend system can easily send data to the user, without needs to implment a specific frontend.

          \item \textbf{Infrastructure.} Implies a virtualized system architecture of a concept and an application stack required for running an application of the respective prototype, based on 3-tier architecture.

          \item \textbf{Tools and Protocols.} Assumes a set of web-based frameworks and Internet of Things(IoT) protocols for real-time data streaming of a wide range of resources used to explore and maintain services of the aforementioned application.
     \end{itemize}


\section{Research Questions and Goals}
       As mentioned in the previous section, there are already exist a lot of solutions for creating sensor-aware applications. But such type of platforms are focused on a single area of usage and they are not commonly suitable to support the dynamic and adaptable composition of different type of data sources in one dashboard.

       Within this thesis the following research questions should be answered in order to design, implement and evaluate a generic frontend for retrieving data sources: 
       \begin{itemize}
       \item Which architecture should have a concept of a generic frontend?
       \item How should it be designed in order to provide ease of integration with any backend system?
       \item What type of data sources have to be retrieved and what is the most universal interface for collaboration between backend and frontend systems?
       \item How can be a generic GUI designed and implemented by providing dynamic content retrieving?
       \item Which protocol can perform real-time data streaming independently from type of data in it?
       \item Which software components might be applied to the concept in order to be applicable to the most available data sources and platform?
       \end{itemize}

     Therefore, this thesis is aimed at the development of a concept that provides users a possibility to personalize their current environment indepently from any type and kind of surrounding sensors. And provide concept of a generic frontend which can be easily interconnected through the supported interface and extensions points with each other.

\section{Structure}

After the introduction in \emph{Chapter 1} the thesis is structured in the following way:

\emph{Chapter 2} defines the background of the master's thesis, describes the basics of used terminologies and the foundation platforms. Requirements to a concept of a generic frontend that has to be developed are also introduced in this chapter.

\emph{Chapter 3} is devoted to the state of the art analysis. Related research and consumer-based works in the area of sensor-based data retrieving approaches, such as: portal and mashup systems, the browser based and non-browser based systems are investigated and evaluated against the defined requirements.

\emph{Chapter 4} focuses on the concept of the generic frontend for exploring sensor and Information services, considering possible approaches, strategies, web frameworks and necessary criteria, defined in \emph{Chapter 3}.

\emph{Chapter 5} provides the implemented functionality of the concept and evaluated by using convinsing scenario. An example of use case scenario are proposed, described and performed in order to cover the fulfillment of the defined requirements by the developed solution. Important methods and extensions of a an interface are described in details.

\emph{Chapter 6} concludes the master's thesis underlining and evaluating achieved goals and providing prospects for the possible future work.  
