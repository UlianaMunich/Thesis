%!TEX root = Thesis.tex

\chapter{Introduction}

  \begin{singlespace}
     The increasing numbers of sensor devices has increased the number of sensor-specific protocols, platforms and software. The definition ``sensor'' means not only a physical device that can be an actuator or measuring device, but also a context-dependent information that is available through the Web, e.g. feeds, tweets, news, weather forecast and other real-time data streaming. As a result, various approaches have been proposed to interconnect, maintain and monitor various types of data sources\cite{6588063,bendel2013service,song2010real}. Those are specifically focused on a platform development, protocol definition and software architecture for the concrete user-oriented requirements. They are typically narrowly focused and do not define a common approach. Mainly in proposed approaches was discovered such type of questions as security and privacy between sensor-specific protocol and platform, online aggregation of a secure data in the Cloud, monitoring of different types of data sources, social dimensions\cite{eggert2013sensorcloud}, where not a single project from the list above does not cover criteria of a user-friendly interface, which is for user nowadays an essential part of opportunity to explore a system. Therefore the area of research of this master thesis is dedicated to define common approach of a generic frontend together with a Graphical User Interface for exploring data sources, that can be easily and universally interconnected to any platform or a system, which has no GUI. Continuously changing information can be retrieved as a real-time streaming, independently from type of a data. Together with an opportunity to get not only an up-to-date value from a data source appear a possibility to have an access to its historical archives. The following sections ground the motivation of the chosen research field, define the central research questions and goals of this master thesis, and describe the overall structure of the work.
  \end{singlespace}

\section{Motivation}
     In the recent years with the technological progress in information systems, web technologies and in particular sensor data systems, have become an essential part in daily life of the modern society. More and more aspects of human life are shifted to the Web and mobile applications. This allows fast and easy way to get information from any point on earth by using any web, mobile, traditional desktop clients. In the same time have increased a range of sensor specific platform and interfaces. People have started to use them more often not only for business, education, manufacture, but also for private reasons(e.g. ``smart house'', ``smart device''). Currently, most of researches are concerned with the protocol and middleware levels, whereas the potential of a generic interactive access to sensor and information services via user-friendly GUI  needs to be reseached. As was already mentioned, most of works in this area are focused mainly only on a backend side, thus an essential goal for this thesis is to provide simple and universal for integration web-based generic frontend with a main focus on a dynamic GUI composition. Users and developers should be able to explore not only information provided by Web services, but also from a real sensors around them. The system architecture of a concept should be highly distributed in order to make possible to interconnect already implemented projects with it.

     Creating composite third-party services and applications from reusable components is an important technique in software engineering and data management. Although, a large body of research and development cover integration at the data and platform levels, weak work has been done to facilitate it at a generic level. This master thesis discusses the existing web frameworks and component technologies used on presentation level integration, illustrates their strengths and weaknesses, and presents some opportunities for future works. Real-time web applications are connection-oriented applications, with web-based user interfaces, that display information as soon as it was published. Examples include social news aggregators and monitoring tools that continually update themselves with data from an external source.

\section{Application Area}
     In order to define concrete research questions, it is important to determine which kind of concept is going to be proposed and implemented, as well as to clarify the common terms which are used throughout the thesis:
     \begin{itemize}
          \item \textbf{Application.} This thesis is focused on a generic multi-tier web-based frontend that delivers dynamic real-time content to end-users via user-friendly graphical interface. Such type of universal frontend acts as middleware between provider-specific data source and a user. Any backend system which supports a concept architecture can easily send data to the user, without needs to implement a narrowly oriented frontend.

          \item \textbf{Infrastructure.} Implies a virtualized system architecture of a concept and an application stack required for running an application and handling real-time data streaming.

          \item \textbf{Tools and Protocols.} Assumes a set of frameworks and Internet of Things(IoT) protocols for real-time data streaming to explore and maintain services of the aforementioned application.
     \end{itemize}


\section{Research Questions and Goals}
       As mentioned in the previous section, there are already exist a lot of solutions for creating sensor-aware applications. But such type of platforms are focused on a single area of usage and they are not commonly suitable to support the dynamic and adaptable composition of different type of data sources in one application.

       Within this thesis the following research questions should be answered in order to design, implement and evaluate a generic frontend for retrieving data sources: 
       \begin{itemize}
       \item Which architecture should have a concept of a generic frontend?
       \item How should it be designed in order to provide ease of integration with any backend system?
       \item What type of data sources have to be retrieved and what is the most universal interface for collaboration between backend and frontend systems?
       \item How can be a generic GUI designed and implemented by providing dynamic content composition?
       \item Which protocol can perform real-time data streaming and access to historical archives of data streams independently from type of data in it and bind it with user personal preferences?
       \item Which software components might be applied to the concept in order to be applicable to the most available data sources and platforms?
       \end{itemize}

     Therefore, this thesis is aimed at the development of a concept that provides to users a possibility to personalize their current environment indepently from a type of a technical configuration of a data source. A concept of a generic frontend can become an essential approach to create project templates for stream-driven applications.

\section{Structure}

After the introduction in \emph{Chapter 1} the thesis is structured in the following way:

\emph{Chapter 2} defines the background of the master's thesis, describes the basics of used terminologies and the foundation platforms. Requirements to a concept of a generic frontend that has to be developed are also introduced in this chapter.

\emph{Chapter 3} is devoted to the state of the art analysis. Related research and consumer-based works in the area of sensor-based data retrieving approaches, such as: portal and mashup systems, the browser based and non-browser based systems are investigated against the defined requirements.

\emph{Chapter 4} focuses on the concept of the generic frontend for exploring sensor and information services, considering possible approaches, strategies, frameworks and necessary criteria, defined in \emph{Chapter 3}.

\emph{Chapter 5} provides the implemented functionality of the concept and evaluated by using convinsing scenario. An example of use case scenario are proposed, described and performed in order to cover the fulfillment of the defined requirements by the developed solution. Important methods and extensions of an interface are described in details.

\emph{Chapter 6} concludes the master's thesis underlining and evaluating achieved goals and providing prospects for the possible future work.  
