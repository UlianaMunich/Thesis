\chapter{Registry JSON Standard}
In the Section 5.3 \emph{Evaluation} was introduced only some part of a Regisry interface standrad. The whole metadata which is needed from sensor in order to dynamically build whole content and functional handlers are shown in the Listing~\ref{hardware_registry}, on the example of ambient temperature sensor. Type of data provided by this sensor is a map of temperature values, which is updated every 5 seconds.
\begin{lstlisting}[label=hardware_registry,caption=JSON Description Format]
[
     {
            "id": "30",
            "title": "Ambient Temperature INF3084",
            "availability": true,
            "last_update": "2014-04-03T11:14:34.000+02:00",
            "description": "DSE students receive an opportunity to get current outdoor temperature in Dresden Momsenstr.20. This sensor provides temperature updates with 5-second frequency and represents a class of low-cost, high-performance sensors. It is implemented using a commercially available Raspberry Pi single-board computer with an affiliated USB thermometer and automatic WLAN and XMPP connections to a sensor MUC room established at boot time.",
            "sla": "Sensor resolution is 0.5 C, measurements taken every 10 seconds. Uptime 95% from 6:00 till 22:00.",
            "sla_last_update":"1395005996",
            "icon": "img/icon/temper_outside.png",
            "preview": [
                "img/sensor_images/preview/id_1_3.png",
                "img/sensor_images/preview/id_1_2.png",
                "img/sensor_images/preview/id_1_1.png"
            ],
            "picture": "img/sensor_images/temper.jpg",
            "access": "private",
            "provider_name": "Provider TU Dresden",
            "location": "Dresden",
            "provider_www": "http://www.inf.tu-dresden.de/",
            "dev_details": "true",
            "responsible_team": "RN",
            "administrator": "Philipp Grubitzsch",
            "EPConfig": "config30.pdf",
            "end_points": [
                {
                    "type": "muc",
                    "name": "xmpp://inf3084@conference.mobilis-dev.inf.tu-dresden.de",
                    "pwd": null
                },
                {
                    "type": "muc",
                    "name": "xmpp://inf3086@conference.mobilis-dev.inf.tu-dresden.de",
                    "pwd": null
                }
            ],
            "type": "chart",
            "template": {
                "subtitle": {
                    "text": "Real-time updated"
                },
                "yAxis": {
                    "title": {
                        "text": "Temperature, C"
                    }
                },
                "series": [
                    {
                        "data": [],
                        "name": "Ambient Temperature INF3084"
                    }
                ],
                "title": {
                    "text": "Ambient Temperature INF3084"
                },
                "chart": {
                    "type": "spline"
                },
                "xAxis": {
                    "labels": {
                        "rotation": -45
                    },
                    "type": "datetime"
                },
                "plotOptions": {
                    "area": {
                        "turboThreshold": 20
                    }
                },
                "credits": {
                    "enabled": false
                }
            }
        }
]
\end{lstlisting}
For software sensor,which produces text messages, a standart metadata descriptions shown in the Listing~\ref{software_registry}
\begin{lstlisting}[label=software_registry,caption=JSON Description Format for Software Sensor]
[
    {
        "id": "20",
        "title": "IT News Feed",
        "availability": true,
        "last_update": "2014-04-02T11:14:34.000+02:00",
        "description": "Provides up to date news in IT and Telecommunication area together with information about new gadgets.",
        "sla": "",
        "sla_last_update": "1395663569",
        "icon": "img/sensor_images/itIcon.jpg",
        "picture": "img/sensor_images/itNews.jpg",
        "preview": "",
        "access": "public",
        "provider_name": "TU Dresden",
        "provider_www": "http://www.inf.tu-dresden.de/",
        "location": "Dresden",
        "dev_details": "true",
        "EPConfig": "config20.pdf",
        "end_points": [
            {
                "type": "muc",
                "name": "xmpp://testraum@conference.mobilis-dev.inf.tu-dresden.de",
                "pwd": null
            }
        ],
        "type": "text",
        "template": ""
    }
]
\end{lstlisting}

\chapter{XEP0049 saving user preferences}
In the Section 5.3.2 \emph{SensDash Implementation} the functionality to save and retrieve personal preferences of a user by using the XMPP private XML-based namespace are shown below in the subscriptions map:
\begin{lstlisting}
{
    "20": [
        "0": {
            "handler_id": "20",
            "handler_type": "text",
            "name": "xmpp://testraum@conference.mobilis-dev.inf.tu-dresden.de",
            "pwd": null,
            "sla_last_update": "",
            "type": "muc"
        }
    ],
    "30": [
        "0": {
            "handler_id": "30",
            "handler_type": "chart",
            "name": "xmpp://inf3084@conference.mobilis-dev.inf.tu-dresden.de",
            "pwd": null,
            "sla_last_update": "1395005996"type: "muc"
        },
        "1": {
            "handler_id": "30",
            "handler_type": "chart",
            "name": "xmpp://chat1@conference.likepro.co",
            "pwd": null,
            "sla_last_update": "1395005996",
            "type": "muc"
        }
    ]
}
\end{lstlisting}
But in comparison to subscriptions map of a user the his/her personal favorites are saved in array, which consists only ids of sensor: ["20", "30"].

\chapter{AngularJS and Reliable Sensor Functionality}
In the Section 5.3.2 \emph{SensDash Implementation} the one of the main functional characteristic of a sensor, namely reliability was implemented.

Firstly, the list of end-points defined in Registry as "end\_points" have to be sorted based on their priority on a server side, example are in the Appendix A. Then, SensDash checks if the room is not empty and also if it has someone except of a user. The part of a code below are shown below:
\begin{lstlisting}
 check_room: function(full_room_name, end_points) {
            if (!xmpp.endpoints_to_handler_map[full_room_name]) {
                return;
            }
            var ep = xmpp.endpoints_to_handler_map[full_room_name];
            if (ep.participants.length == 0) {
                console.log("Room is empty, endpoint rejected: " + full_room_name);
                for (var i = 0; i < end_points.length; i++) {
                    // find invalid room in all end_points list, delete it, and call subscribe again
                    if (end_points[i].handler_id == ep.handler_id) {
                        xmpp.unsubscribe(end_points[i]);
                        end_points.splice(i, 1);
                        xmpp.subscribe(end_points);
                    }
                }
            } else {
                console.log("Room is not empty, endpoint approved: " + full_room_name);
            }
        }
\end{lstlisting}
This functionality enhance the basic list of xmpp-based functions and called for xmpp subscriptions(joining the room).

\chapter{AngularJS factory for XMPP Connection}
In the Section 5.3.2 \emph{SensDash Implementation} in order to handle all requests which goes through the XMPP, was implemented next skeleton:
\begin{lstlisting}
sensdash_services.factory("XMPP", ["$location", "Graph", "Text", function ($location, Graph, Text) {
    if (typeof Config === "undefined") {
        console.log("Config is missing or broken, redirecting to setup reference page");
        $location.path("/reference");
    }
    var BOSH_SERVICE = Config.BOSH_SERVER;
    var PUBSUB_SERVER = Config.PUBSUB_SERVER;
    var PUBSUB_NODE = Config.PUBSUB_NODE;
    var xmpp = {
        connection: {connected: false},
        endpoints_to_handler_map: {},
        received_message_ids: [],
        // logging IO for debug
        raw_input: function (data) {
            console.log("RECV: " + data);
          },
        // logging IO for debug
         raw_output: function (data) {
             console.log("SENT: " + data);
         },
        connect: function (jid, pwd, callback) {
            xmpp.connection = new Strophe.Connection(BOSH_SERVICE);
            xmpp.connection.connect(jid, pwd, callback);
        //    xmpp.connection.rawInput = xmpp.raw_input;
        //    xmpp.connection.rawOutput = xmpp.raw_output;
        },
        subscribe: function (end_point, on_subscribe) {
            xmpp.endpoints_to_handler_map[end_point.name] = end_point;
            var jid = xmpp.connection.jid;
            if (end_point.type == "pubsub") {
                xmpp.connection.pubsub.subscribe(
                    jid,
                    PUBSUB_SERVER,
                    PUBSUB_NODE + "." + sensor_map,
                    [],
                    xmpp.handle_incoming_pubsub,
                    on_subscribe);
                console.log("Subscription request sent,", end_point);
            } else if (end_point.type == "muc") {
                var nickname = jid.split("@")[0];
                var room = end_point.name.replace("xmpp://",'');
                xmpp.connection.muc.join(room, nickname, xmpp.handle_incoming_muc);
                on_subscribe();
            } else {
                console.log("End point protocol not supported");
            }
        },
        unsubscribe: function (end_point, on_unsubscribe) {
            var jid = xmpp.connection.jid;
            if (end_point.type == "pubsub") {
                xmpp.connection.pubsub.unsubscribe(PUBSUB_NODE + "." + end_points);
                on_unsubscribe();
            } else if (end_point.type == "muc") {
                var room = end_point.name.replace("xmpp://",'');
                xmpp.connection.muc.leave(room, jid.split("@")[0]);
                on_unsubscribe();
            }
        },
        find_sensor: function (message) {
            var endpoint_name = message.getAttribute('from');
            //to get to the end_points.name and add "xmpp://"
            endpoint_name = "xmpp://" + endpoint_name.replace(/\/\w+/g, '');
            if (endpoint_name in xmpp.endpoints_to_handler_map) {
                var handler_id = xmpp.endpoints_to_handler_map[endpoint_name].handler_id;
                var handler_type = xmpp.endpoints_to_handler_map[endpoint_name].handler_type;
                return {"id": handler_id, "type": handler_type};
            } else {
                // endpoint not found in subscriptions map
                return false;
            }
        },
        handle_incoming_muc: function (message) {
            var sensor = xmpp.find_sensor(message);
            if (!(sensor)) {
                // sensor not found
                return true;
            }
            var text = Strophe.getText(message.getElementsByTagName("body")[0]);
            if (sensor.type == 'text') {
                if (typeof text == "string") {
                    Text.updateTextBlock(text, sensor["id"]);
                } else {
                    console.log("Message is not a Text");
                }
            } else if (sensor.type == 'chart') {
                try {
                    text = text.replace(/&quot;/g, '"');
                    var msg_object = JSON.parse(text);
                    var data_array = [];
                    console.log("JSON message parsed: ", msg_object);
                    //creating a new array from received map for Graph.update in format [timestamp, value], e.g. [1390225874697, 23]
                    if ('sensorevent' in msg_object) {
                        var time_UTC = msg_object.sensorevent.timestamp;
                        var time_UNIX = new Date(time_UTC).getTime();
                        data_array[0] = time_UNIX;
                        data_array[1] = msg_object.sensorevent.values[0];
                    } else {
                        data_array = msg_object;
                    }
                    console.log(data_array);
                } catch (e) {
                    console.log("message is not valid JSON", text);
                    return true;
                }
                if (Array.isArray(data_array)) {
                    Graph.update(data_array, sensor.id);
                }
            }
            return true;
        },
        handle_incoming_pubsub: function (message) {
            if (!xmpp.connection.connected) {
                return true;
            }
            var server = "^" + Client.pubsub_server.replace(/\./g, "\\.");
            var re = new RegExp(server);
            if ($(message).attr("from").match(re) && (xmpp.received_message_ids.indexOf(message.getAttribute("id")) == -1)) {
                xmpp.received_message_ids.push(message.getAttribute("id"));
                var _data = $(message).children("event")
                    .children("items")
                    .children("item")
                    .children("entry").text();
                var _node = $(message).children("event").children("items").first().attr("node");
                var node_id = _node.replace(PUBSUB_NODE + ".", "");

                if (_data) {
                    // Data is a tag, try to extract JSON from inner text
                    console.log("Data received", _data);
                    var json_obj = JSON.parse($(_data).text());
                    Graph.update(json_obj, node_id);
                }
            }
            return true;
        }
    };
    return xmpp;
}]);
\end{lstlisting}
Which has main functions: XMPP.connect(), XMPP.subscribe()(In case of MUC chat room it means join the room), XMPP.unsubscribe()(For MUC chat rooms it means leave the room), XMPP.handle\_incoming\_muc(), XMPP.handle\_incoming\_pubsub(). Last two functions handles MUC or PubSub connection, based on the type of end\_point, given in a Registry.